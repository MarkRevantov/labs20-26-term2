\documentclass[10pt, a5paper]{book}

\usepackage[a5paper, top=1cm, bottom=0.5cm,]{geometry}

\usepackage{soulutf8}
\usepackage[utf8]{inputenc}
\usepackage[russian]{babel}
\usepackage{amsmath}
\usepackage{amssymb}
\begin{document}
\setcounter{page}{219}
сается образа непрерывного линейного оператора, то он не~обязательно будет подпространством в~$E_1$, даже если~$D_A = E$.

Понятие линейного функционала, введенное в~начале этой главы, есть частный случай линейного оператора. Именно, линейный функионал~--- это линейный~оператор, переводящий данное пространство $E$ в~числовую прямую $R^1$. Определение линейности и~непрерывности оператора переходят при~$E_1 = R^1$ в~соответствующие определения, введенные ранее для функционалов.

Точно так~же и~ряд дальнейших понятий и~фактов, излагаемых ниже для~линейных операторов, представляет собой довольно автоматическое обобщение результатов, уже~изложенных в \S~1~этой~главы применительно к линейный функционалам.

\so{Примеры линейных операторов}. 1. Пусть $E$~--- линейное топологическое пространство. Положим
$$lx = x \quad \text{для всех}~x\in~E$$

Такой оператор, переводящий каждый элемент пространства в~себя, называется {\it единичным оператором}.

2.~Пусть $E$ и $E_1$~--- произвольные линейные топологические пространства и пусть
$$Ox=0 \quad \text{для всех}~x\in~E$$ 
(Здесь $0$~--- нулевой элемент пространства $E_1$). Тогда $O$ назы\-вается {\it нулевым оператором}.

3.{\it~Общий вид линейного оператора, переводящего конечно\-мерное пространство в конечномерное.} Пусть $A$~--- линейный оператор, отображающий n-мерное пространство $R^n$ с~базисом $e_1,\dots,e_n$ в~$m$-мерное пространство $R^m$ с~базисом $f_1,\dots, f_m$. Eсли~$x$~--- произвольный вектор из $R^n$, то
$$x=\sum^n_{i=1}~x_i e_i$$
и~в~силу линейности оператора $A$
$$Ax=\sum^n_{i=1}~x_i A e_i$$
Таким образом, оператор $A$ задан, если известно, во~что~он~переводит базисные векторы $e_1,\dots, e_n$. Рассмотрим разложения векторов $Ae_i$ по базису $f_1,\dots,f_m$. Имеем
$$A e_i = \sum^m_{k=1} a_{ki} f_k .$$
Отсюда ясно, что оператор $A$ определяется матрицей коэффициентов $\|a_{ki}\|$. Образ пространства $R^n$ в $R^m$ представляет собой линейное подпространство, размерность которого равна, очевидно, рангу матрицы $\|a_{ki}\|$,~т.~е. во~всяком случае, непревосходит $n$. Отметим, что всякий линейный оператор, заданный в~конечномерном пространстве, автоматически непрерывен.

4. Рассмотрим шильбертово пространство $H$ и в~нем некоторое подпространство $H_1$. Разложив $H$ в~прямую сумму подпространства $H_1$ и его~ортогонального дополнения, т.~е. представив каждый элемент $h\in H$ в~виде
$$h = h_1 + h_2 \quad (h_1\in H_1, h_2 \perp H_1),$$
положим $Ph = h_1$. Этот~оператор $P$ естественно назвать {\it оператором ортогонального проектирования}, или{\it ортопроектором} $H$ на~$H_1$. Линейность и~непрерывность проверяются без~труда.

5. Рассмотрим в~пространстве непрерывных функций на~отрезке $[a,b]$ оператор, определяемый формулой
$$\psi(s) = \int\limits_a^b K(s, t) \phi(t) dt, \eqno{(1)}$$
где $K(s,t)$~--- некоторая фиксировання непрерывная функция двух переменных. Функция $\psi(s)$ непрерывна для~любой непрерывной функции $\psi(t)$, так~что оператор $(1)$ действительно переводит пространство непрерывных функций в~себя. Его~линейность очевидна. Для~того чтобы говорить о~его непрерывности, необходимо предварительно указать, какая топология рассматривается в~нашем пространстве непрерывных функций. Читателю предлагается доказать непрерывность оператора в~случаях, когда: а)~рассматривается пространство $C[a,b]$, т.~е. пространство непрерывных функций с нормой $\|\phi\|=\max |\phi(t)|$; б)~когда рассматривается $C_2[a,b]$, т.~е. $\|\phi\| = \left(\int\limits_a^b \phi^2(t) dt\right)^{\frac{1}{2}}.$

6. В~том~же пространстве непрерывных функций рассмотрим оператор
$$\psi(t) = \phi_0(t)\phi(t),$$
где $\phi_0(t)$~--- фиксированная непрерывная функция. Линейность этого оператора очевидна. (Докажите его непрерывность при нормировках, указанных в~предыдущем примере.)

7. Один из~важнейших для анализа примеров линейных операторов~--- это оператор дифференцирования. Его можно рассматривать в~различных пространствах.

а) Рассмотрим пространство непрерывных функций $C[a,b]$ и оператор
$$Df(t)=f'(t),$$

\end{document}